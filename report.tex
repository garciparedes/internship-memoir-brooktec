\documentclass[10pt, a4paper,spanish]{article}
\usepackage[utf8]{inputenc}

\usepackage{varwidth}
\usepackage{graphicx}

\usepackage[T1]{fontenc} % Use 8-bit encoding that has 256 glyphs
\usepackage{microtype} % Slightly tweak font spacing for aesthetics

\usepackage[hmarginratio=1:1,top=32mm,columnsep=20pt]{geometry} % Document margins
\usepackage[hang, small,labelfont=bf,up,textfont=it,up]{caption} % Custom captions under/above floats in tables or figures
\usepackage{booktabs} % Horizontal rules in tables
\usepackage{float} % Required for tables and figures in the multi-column environment - they need to be placed in specific locations with the [H] (e.g. \begin{table}[H])
\usepackage{hyperref} % For hyperlinks in the PDF

\usepackage{lettrine} % The lettrine is the first enlarged letter at the beginning of the text
\usepackage{paralist} % Used for the compactitem environment which makes bullet points with less space between them

\usepackage{fancyhdr} % Headers and footers
\pagestyle{fancy} % All pages have headers and footers
\fancyhead{} % Blank out the default header
\fancyfoot{} % Blank out the default footer
\fancyhead[C]{ Mayo 2016 $\bullet$ JumpVa $\bullet$ Informe Final} % Custom header text
\fancyfoot[RO,LE]{\thepage} % Custom footer text

%----------------------------------------------------------------------------------------
%	TITLE SECTION
%----------------------------------------------------------------------------------------

\title{\vspace{-15mm}\fontsize{24pt}{10pt}\selectfont\textbf{Informe Final}} % Article title

\author{
\large
\textsc{Sergio García Prado}\\[2mm] % Your name
\normalsize Universidad de Valladolid \\ % Your institution
\vspace{-5mm}
}
\date{\today}

%----------------------------------------------------------------------------------------

\begin{document}
	\begin{titlepage}
	\centering
		%\includegraphics[width=0.15\textwidth]{example-image-1x1}\par\vspace{1cm}
		{\scshape\LARGE Universidad de Valladolid \par}
		\vspace{1cm}
		{\scshape\Large Brooktec S.L.\par}
		\vspace{1.5cm}
		{\huge\bfseries Prácticas en Empresa\par}
		\vspace{2cm}
		{\large
		\textsc{Sergio García Prado\textsubscript}\\[2mm] % Your name
		\vspace{-5mm}
		}

		\vfill

	% Bottom of the page
		{\large \today\par}
	\end{titlepage}

	\thispagestyle{fancy} % All pages have headers and footers

%----------------------------------------------------------------------------------------
%	TABLE OF CONTENTS
%----------------------------------------------------------------------------------------

	\tableofcontents
	\newpage

%----------------------------------------------------------------------------------------
%	TEXT
%----------------------------------------------------------------------------------------



    \section{Datos Generales de la Práctica}



    \section{Breve Descripción de la Empresa}



    \section{Memoria de Actividades}

        \paragraph{}
        Durante mi estancia en la empresa trabajé en varios projectos a la vez que pasé una fase inicial de formación en la cual me familiaricé con las tecnologías que utilizaría más adelante. Las diferentes etapas a veces no están muy bien diferenciadas en la linea temporal ya que hubo proyectos en los que participé al principio y luego se me asignaron otras pequeñas tareas ajenas a ello para después volver a retomar el proyecto.

        \paragraph{}
        Exceptuando la etapa de aprendizaje, en la cual mayoritariamente era yo el encargado de qué hacer (siempre bajo unas guías sobre en qué tecnologías debería adquirir soltura), en el resto de proyectos el patrón de trabajo era el siguiente: Una pequeña explicación acerca de la actividad que debería desempeñar junto con ayuda para configurar el entorno de desarrollo y un conjunto de diseños en los cuales me debía basar para realizar el desarrollo. A partir de este punto mis compañeros me resolvían dudas en el caso de que las tuviera y una vez finalizaba la tarea estos la revisaban antes de subirla al entorno de producción.

        \subsection{Aprendizaje}


        \subsection{Nueva Web de SocialNoise}


        \subsection{Cambios en Webtogo}


        \subsection{Nueva sección Descubre ERL de El Rey León}


        \subsection{Nueva Web de Legal Lifeline}



    \section{Conclusiones}



    \newpage
    \section{Declaración de Responsabilidad}

        \includegraphics[width=\textwidth]{res/responsibility-declaration}



\end{document}
